\documentclass[../course]{subfiles}

\begin{document}

\chapter {Discrete Fourier Transform}

\subfile{syllabus.tex}

\section {Introduction}


The DFT $X[k]$ of a finite length sequence $x[n]$ defined for
$n = 0 \ldots N-1$ can be obtained by sampling its DTFT,

\begin{align}
    X(e^{j\omega})
\end{align}

on the $\omega$ axis between $0 \leq \omega < 2 \pi$ at

\begin{align}
    \omega_{k} = \dfrac{2 \pi k}{N}
    \; \bigg|_{ k = 0 \ldots N - 1 }
\end{align}

ie,

\begin{align}
    \mbox{ DFT }\{x[n]\} &= X[k] \\
    &= X(e^{ j \omega }) \; \bigg|_{ \omega = \dfrac{ 2 \pi k }{ N }} \\
    &= \sum_{ n = 0 }^{ N - 1 } x[n] \; e^{\bigg( - \dfrac{ j 2 \pi k n }{ N } \bigg)}
\end{align}

Using the commonly used notation,

\begin{align}
    \mathbf{X[k]} =
    \sum_{ n = 0 }^{ N - 1 } \mathbf{x[n]} \; \mathbf{W_{N}}^{kn}
    \; \bigg|_{ k = 0 \ldots N - 1 }
\end{align}

Where:

\begin{itemize} [label=]
    \item $x[n]$: is the sequence
    \item $X[K]$: is the DFT
    \item $W_{N}$: is

        \begin{align}
            W_{N} = e^{\bigg(\dfrac{ - j 2 \pi }{N}\bigg)}
        \end{align}

\end{itemize}

\subsection{DFT Equation as a Linear Transform}

It is possible to view the DFT equation as a linear transformation on the
sequence $x[n]$ as,

\begin{align}
    \mathbf{ X = D_{N} \cdot x }
\end{align}

Where:

\begin{itemize} [label=]

    \item $\mathbf{X}$: is the vector composed of $N$ DFT samples,

        \begin{align}
            X = \big[
                X[0], X[1], \ldots X[N - 1]
                \big]^{T}
        \end{align}

    \item $\mathbf{x}$: is the vector of $N$ input samples,

        \begin{align}
            x = \big[
                x[0], x[1], \ldots x[N - 1]
                \big]^{T}
        \end{align}

    \item $\mathbf{D_{N}}$: is the $N \times N$ DFT matrix given by,

        \begin{align}
            D_{N} =
            \begin{pNiceMatrix}[columns-width = auto]
                1      &               1 &               1 & \ldots &                      1 \\
                1      &       W_{N}^{1} &       W_{N}^{2} & \ldots &        W_{N}^{(N - 1)} \\
                1      &       W_{N}^{2} &       W_{N}^{4} & \ldots &      W_{N}^{2 (N - 1)} \\
                \vdots &          \vdots &          \vdots &  \cdot &                 \vdots \\
                1      & W_{N}^{(N - 1)} & W_{N}^{(N - 1)} & \ldots & W_{N}^{(N - 1)(N - 1)} \\
            \end{pNiceMatrix}
        \end{align}

\end{itemize}

To understand how this is correct, consider a general linear transformation of
the form $y = Ax$. We can write this matrix equation in scalar form as $y_{i} =
\sum_{j} a_{ij} x_{j}$. If we rearrange the DFT equation as,

\begin{align}
    X[k] =
    \sum_{n = 0}^{N - 1}
    W_{N}^{kn} \; x[n]
\end{align}

we see that the two are similar in form. Just as the $(i, j)\mbox{th}$ element
of $A$ is $a_{ij}$, the $(k, n)\mbox{th}$ element (counting from 0) of $D_{N}$
is $W_{N}^{kn}$. This is exactly how the matrix $D_{N}$ is arranged.

\section{Computing DFT Matrix Using Loops}

We can generate the DFT matrix in a straightforward manner using two for loops:

\subsection{Python Implementation using Two Loops}

%python/dftMatrixTwoLoopMethod.py%
\begin{minted} [breaklines, autogobble] {python}
    import numpy as np
    import matplotlib.pyplot as plt

    N = 4
    D = np.empty((N, N), dtype = np.cdouble)

    W = np.exp(-1j * 2 * np.pi / N)

    for k in np.arange(N):
    for n in np.arange(N):
    D[k, n] = W ** (k * n)

    np.round(D)
    print(D)
\end{minted}

\paragraph{Output}

\begin{minted}[breaklines, autogobble] {text}
    [[ 1.0000000e+00+0.0000000e+00j  1.0000000e+00+0.0000000e+00j
       1.0000000e+00+0.0000000e+00j  1.0000000e+00+0.0000000e+00j]
     [ 1.0000000e+00+0.0000000e+00j  6.1232340e-17-1.0000000e+00j
      -1.0000000e+00-1.2246468e-16j -1.8369702e-16+1.0000000e+00j]
     [ 1.0000000e+00+0.0000000e+00j -1.0000000e+00-1.2246468e-16j
       1.0000000e+00+2.4492936e-16j -1.0000000e+00-3.6739404e-16j]
     [ 1.0000000e+00+0.0000000e+00j -1.8369702e-16+1.0000000e+00j
      -1.0000000e+00-3.6739404e-16j  5.5109106e-16-1.0000000e+00j]]
\end{minted}

\subsection{Python Implementation Using a Single Loop}

One for loop can be eliminated if we define \mintinline[]{python}{k} as an array:

%python/dftMatrixOneLoopMethod.py%
\begin{minted}[breaklines, autogobble] {python}
    import numpy as np
    import matplotlib.pyplot as plt

    N = 4
    D = np.empty((N, N), dtype = np.cdouble)

    W = np.exp(-1j * 2 * np.pi / N)
    k = np.arange(N)

    for n in np.arange(N):
        D[:, n] = W ** (k * n)

    np.round(D)
    print(D)
\end{minted}

\paragraph{Output}

\begin{minted}[breaklines, autogobble] {text}
    [[ 1.0000000e+00+0.0000000e+00j  1.0000000e+00+0.0000000e+00j
       1.0000000e+00+0.0000000e+00j  1.0000000e+00+0.0000000e+00j]
     [ 1.0000000e+00+0.0000000e+00j  6.1232340e-17-1.0000000e+00j
      -1.0000000e+00-1.2246468e-16j -1.8369702e-16+1.0000000e+00j]
     [ 1.0000000e+00+0.0000000e+00j -1.0000000e+00-1.2246468e-16j
       1.0000000e+00+2.4492936e-16j -1.0000000e+00-3.6739404e-16j]
     [ 1.0000000e+00+0.0000000e+00j -1.8369702e-16+1.0000000e+00j
      -1.0000000e+00-3.6739404e-16j  5.5109106e-16-1.0000000e+00j]]
\end{minted}

\section{Computing DFT Matrix Using Matrix Method}

Compute the DFT of the sequence $x[n] = {1, 2, 3, 4}$ using the matrix method.
And verify the answer using the \mintinline[]{python}{scipy.fft.fft()} function
which implements the fast FFT algorithm for DFT computation:

\subsection{Python Implementation}

%python/dftMatrixMatrixMethod.py%
\begin{minted}[breaklines, autogobble] {python}
    import numpy as np
    import matplotlib.pyplot as plt
    from scipy import fft

    N = 4
    D = np.empty((N, N), dtype = np.cdouble)

    W = np.exp(-1j * 2 * np.pi / N)
    k = np.arange(N)

    for n in np.arange(N):
        D[:, n] = W ** (k * n)

    np.round(D)

    x = np.array([[1, 2, 3, 4]]).T
    X = D @ x

    np.round(X)

    print("DFT using matrix method:")
    print(X)
    print("")

    # verifying using scipy.fft.fft()

    SX = fft.fft(x, axis = 0)

    print("DFT using scipy:")
    print(SX)
\end{minted}

\paragraph{Output}

\begin{minted}[breaklines, autogobble] {text}
    DFT using matrix method:
    [[10.+0.00000000e+00j]
     [-2.+2.00000000e+00j]
     [-2.-9.79717439e-16j]
     [-2.-2.00000000e+00j]]

    DFT using scipy:
    [[10.-0.j]
     [-2.+2.j]
     [-2.-0.j]
     [-2.-2.j]]
\end{minted}

\section{Computing DFt Matrix Using Direct Evaluation of DFT Equation}

Compute the DFT of the sequence in above section by direct evaluation of the
DFT equation.

\subsection{Python Implementation}

%python/dftMatrixEquationMethod.py%
\begin{minted}[breaklines, autogobble] {python}
    import matplotlib.pyplot as plt
    import numpy as np

    global np

    N = 4

    def dft_func (k, n):
        global N
        return np.exp(-1j * 2 * np.pi / N * k * n)

    D = np.empty((N, N), dtype = np.cdouble)

    for k in range(N):
        for n in range(N):
            D[k, n] = dft_func(k, n)

    np.round(D)

    print(D)
\end{minted}

\paragraph{Output}


\begin{minted}[breaklines, autogobble] {text}
    [[ 1.0000000e+00+0.0000000e+00j  1.0000000e+00+0.0000000e+00j
       1.0000000e+00+0.0000000e+00j  1.0000000e+00+0.0000000e+00j]
     [ 1.0000000e+00+0.0000000e+00j  6.1232340e-17-1.0000000e+00j
      -1.0000000e+00-1.2246468e-16j -1.8369702e-16+1.0000000e+00j]
     [ 1.0000000e+00+0.0000000e+00j -1.0000000e+00-1.2246468e-16j
       1.0000000e+00+2.4492936e-16j -1.0000000e+00-3.6739404e-16j]
     [ 1.0000000e+00+0.0000000e+00j -1.8369702e-16+1.0000000e+00j
      -1.0000000e+00-3.6739404e-16j  5.5109106e-16-1.0000000e+00j]]
\end{minted}

\section{Comparing The Computation Time for Each Method}

%TODO: either create a funcion to execute each in a loop, or just copy the%
%output from Jupyter%

Compare the computation time for the three methods using the
\mintinline[]{text}{\%timeit} cell magic in Jupyter notebook.

\subsection{Python Implementation using Jupyter}

%python/dftMatrixTimingJupyter.py%
\begin{minted}[breaklines, autogobble] {python}
    # pass
\end{minted}

\subsection{Python Implementation using timeit}

%python/dftMatrixTimingTimeit.py%
\begin{minted}[breaklines, autogobble] {python}
    # pass
\end{minted}

\section{Computing a 64 Point DFT}

Compute a $64$ point DFT matrix and plot it's real and imaginary parts using
\mintinline[]{text}{plt.imshow(D.real)} and
\mintinline[]{text}{plt.imshow(D.imag)}.

\subsection{Python Implementation}

%python/dftMatrixSixtyFourPoint.py%
\begin{minted}[breaklines, autogobble] {python}
    import numpy as np
    import matplotlib.pyplot as plt

    N = 64
    D = np.empty((N, N), dtype = np.cdouble)

    W = np.exp(-1j * 2 * np.pi / N)
    k = np.arange(N)

    for n in np.arange(N):
        D[:, n] = W ** (k * n)

    np.round(D)
    plt.imshow(D.real)
    plt.savefig("../plots/dftMatrixSixtyFourPointReal.pdf", bbox_inches = "tight")

    plt.clf()
    plt.imshow(D.imag)
    plt.savefig("../plots/dftMatrixSixtyFourPointImaginary.pdf", bbox_inches = "tight")
\end{minted}

\paragraph{Plots}

\begin{center}
    \adjustbox{max width = 0.7\textwidth} {
        \includegraphics[height = 0.8\textheight] {plots/dftMatrixSixtyFourPointReal.pdf}
    }
    \captionof {figure} {Real part of 64 point dft matrix}
\end{center}

\begin{center}
    \adjustbox{max width = 0.7\textwidth} {
        \includegraphics[height = 0.8\textheight] {plots/dftMatrixSixtyFourPointImaginary.pdf}
    }
    \captionof {figure} {Imaginary part of 64 point dft matrix}
\end{center}

\section{Circular Convolution Using Concentric Circles}

The circular convolution of two $N$ point sequences $g[n]$ and h[n] is another
$N$ point sequence $y[n]$ defined as,

\begin{align}
    y[n]
    \sum_{k = 0}^{N - 1} g[k] h\big[(n - k)_{N}\big]
\end{align}

To perform circular convolution graphically, $N$ samples $g[n]$ are equally
spaced around the outer circle in the clockwise direction, and $N$ samples of
$h[n]$ are displayed on the inner circle in the counterclockwise direction
starting at the same point. Corresponding samples on the two circles are
multiplied, and the products are summed to form an output. The successive
value of the circular convolution is obtained by rotating the inner circle of
the one sample in the clockwise direction, and repeating the operation of
computing the sum of corresponding products. This process is repeated until
the first sample of inner circle lines up with the first sample of the
exterior circle again.

\subsection{Python Implementation}

%python/circConvWithoutPadding.py%
\begin{minted}[breaklines, autogobble] {python}
    import numpy as np

    def circonv(g, h):

        if g.size != h.size:
            raise Exception("Sequences not of same length")

        N = g.size
        htr = np.concatenate([[h[0]], h[:0:-1]])

        y = np.zeros(N)

        for n in np.arange(N):
            y[n] = np.sum(g * htr)
            htr = np.roll(htr, 1)

        return y
\end{minted}

\section{Above with Unequally Sized Functions}

The above implementation raises an exception when the sequences are not of the
same length. Modify the function such that if the two sequences are not of the
same length, the shorter sequence is padded with zeros and circular convolution
is computed.

\subsection{Python Implementation}

%python/circConvWithPadding.py%
\begin{minted}[breaklines, autogobble] {python}
    import matplotlib.pyplot as plt
    import numpy as np

    def circonv(g, h):

        if g.size > h.size:
            h = np.concatenate([h, np.zeros(g.size - h.size)])
        elif g.size < h.size:
            g = np.concatenate([g, np.zeros(h.size - g.size)])

        N = g.size
        htr = np.concatenate([[h[0]], h[:0:-1]])

        y = np.zeros(N)

        for n in np.arange(N):
            y[n] = np.sum(g * htr)
            htr = np.roll(htr, 1)

        return y, g, h
\end{minted}

\section{More Circular Convolution}

Using your function, compute the circular convolution of two sequences $g[n] =
\{1, 2, 3, 4, 5\}$ and $h[n] = \{2, 2, 0, 1, 1\}$ and

\subsection{Python Implementation}

%python/moreCircConvComp.py%
\begin{minted}[breaklines, autogobble] {python}
    import matplotlib.pyplot as plt
    import numpy as np

    def circonv(g, h):

        if g.size > h.size:
            h = np.concatenate([h, np.zeros(g.size - h.size)])
        elif g.size < h.size:
            g = np.concatenate([g, np.zeros(h.size - g.size)])

        N = g.size
        htr = np.concatenate([[h[0]], h[:0:-1]])

        y = np.zeros(N)

        for n in np.arange(N):
            y[n] = np.sum(g * htr)
            htr = np.roll(htr, 1)

        return y, g, h

    g = np.array([1, 2, 3, 4, 5])
    h = np.array([2, 2, 0, 1, 1])

    y, g, h = circonv(g, h)

    print("g[n] = ", end = "")
    print(g)

    print("h[n] = ", end = "")
    print(h)

    print("y[n] = ", end = "")
    print(y)

\end{minted}

\paragraph{Output}

\begin{minted}[breaklines, autogobble] {text}
    g[n] = [1 2 3 4 5]
    h[n] = [2 2 0 1 1]
    y[n] = [17. 13. 19. 20. 21.]
\end{minted}

%% \subsubsection{Plots}
%%
%% \begin{center}
%%     \adjustbox{max width = 0.7\textwidth} {
%%         \includegraphics[height = 0.8\textheight] {plots/moreCircConvComp.pdf}
%%     }
%%     \captionof {figure} {More Functions and Their Convolution}
%% \end{center}

\section{Verifying Circular Convolution with Property}

If $y[n]$ is the $N$ point circular convolution of the sequences $g[n]$ and
$h[n]$ defined as $y[n] = g[n] \circledcirc h[n]$, then $Y[k] = G[k] \cdot
H[k]$, where $Y[k]$, $G[k]$, $H[k]$ are the $N$ point DFTs of $y[n]$, $g[n]$
and $h[n]$. We can use this property to compute circular convolution as:

\begin{align}
    y[n] = \mbox{IDFT}\big\{ G[k] \cdot H[k] \big\}
\end{align}

Use this property to verify your circular convolution result in the previous
part. Use \mintinline[]{text}{fft.fft()} and \mintinline[]{text}{fft.ifft()}
functions in the \mintinline[]{text}{scipy()} module to compute the DFTs and
IDFT.

\subsection{Python Implementation}

%python/circConvWithProperty.py%
\begin{minted}[breaklines, autogobble] {python}
    import numpy as np
    from scipy import fft

    def circonv(g, h):

        if g.size > h.size:
            h = np.concatenate([h, np.zeros(g.size - h.size)])
        elif g.size < h.size:
            g = np.concatenate([g, np.zeros(h.size - g.size)])

        N = g.size
        htr = np.concatenate([[h[0]], h[:0:-1]], dtype = np.cdouble)

        y = np.zeros(N, dtype = np.cdouble)

        for n in np.arange(N):
            y[n] = np.sum(g * htr)
            htr = np.roll(htr, 1)

        return y, g, h

    g = np.array([1, 2, 3, 4, 5], dtype = np.cdouble)
    h = np.array([2, 2, 0, 1, 1], dtype = np.cdouble)

    y, g, h = circonv(g, h)

    G = fft.fft(g, axis = 0)
    H = fft.fft(h, axis = 0)

    Y = G * H

    sy = fft.ifft(Y)

    print("Convolution obtained through performing circular convolution:" )
    print(y)

    print("")

    print("Convolution obtained through property:" )
    print(sy)
\end{minted}

\paragraph{Output}

\begin{minted}[breaklines, autogobble] {text}
    Convolution obtained through performing circular convolution:
    [17.+0.j 13.+0.j 19.+0.j 20.+0.j 21.+0.j]

    Convolution obtained through property:
    [17.+0.j 13.+0.j 19.+0.j 20.+0.j 21.+0.j]
\end{minted}

\section{Circular Convolution Using Matrix Method}

The $N$ point circular convolution operation can be written in matrix form as

\begin{align}
    \begin{pNiceMatrix}[columns-width=auto]
        y[0]     \\
        y[1]     \\
        y[2]     \\
        \vdots   \\
        y[N - 1] \\
    \end{pNiceMatrix}
    =
    \begin{pNiceMatrix}[columns-width = 1.7cm]
        h[0]     & h[N - 1] &  [N - 2] & \ldots &   h[1] \\
        h[0]     &     h[0] &  [N - 1] & \ldots &   h[2] \\
        h[0]     &     h[1] &     h[0] & \ldots &   h[3] \\
        \vdots   &  \vdots  &   \vdots &  \cdot & \vdots \\
        h[N - 1] & h[N - 2] & h[N - 3] & \ldots &   h[0] \\
    \end{pNiceMatrix}
    \begin{pNiceMatrix}
        g[0]     \\
        g[1]     \\
        g[2]     \\
        \vdots   \\
        g[N - 1] \\
    \end{pNiceMatrix}
\end{align}

The elements in each row of the matrix above are obtained by circularly
rotating the elements of the previous row to the right by $1$ position. Such a
matrix is called \emph{circulant matrix}. A \emph{circulant matrix} can be
generated using \texttt{scipy.linalg.circulant(c)}. The argument \texttt{c} is
the first column of the matrix. Verify the circular convolution result using
the matrix method.

\subsection{Python Implementation}

%python/circConvWithMatrix.py%
\begin{minted}[breaklines, autogobble] {python}
    import numpy as np
    from scipy.linalg import circulant

    def circonv(g, h):

        if g.size > h.size:
            h = np.concatenate([h, np.zeros(g.size - h.size)])
        elif g.size < h.size:
            g = np.concatenate([g, np.zeros(h.size - g.size)])

        N = g.size
        htr = np.concatenate([[h[0]], h[:0:-1]], dtype = np.cdouble)

        y = np.zeros(N, dtype = np.cdouble)

        for n in np.arange(N):
            y[n] = np.sum(g * htr)
            htr = np.roll(htr, 1)

        return y, g, h

    g = np.array([1, 2, 3, 4, 5], dtype = np.cdouble)
    h = np.array([2, 2, 0, 1, 1], dtype = np.cdouble)

    y, g, h = circonv(g, h)

    circ_mat = circulant(h)

    my = circ_mat @ g

    print("Circulant matrix of h[n]:")
    print(circ_mat)

    print("")

    print("Convolution obtained through performing circular convolution:" )
    print(y)

    print("")

    print("Convolution obtained through matrix method:" )
    print(my)
\end{minted}

\paragraph{Output}

\begin{minted}[breaklines, autogobble] {text}
    Circulant matrix of h[n]:
    [[2.+0.j 1.+0.j 1.+0.j 0.+0.j 2.+0.j]
     [2.+0.j 2.+0.j 1.+0.j 1.+0.j 0.+0.j]
     [0.+0.j 2.+0.j 2.+0.j 1.+0.j 1.+0.j]
     [1.+0.j 0.+0.j 2.+0.j 2.+0.j 1.+0.j]
     [1.+0.j 1.+0.j 0.+0.j 2.+0.j 2.+0.j]]

    Convolution obtained through performing circular convolution:
    [17.+0.j 13.+0.j 19.+0.j 20.+0.j 21.+0.j]

    Convolution obtained through matrix method:
    [17.+0.j 13.+0.j 19.+0.j 20.+0.j 21.+0.j]
\end{minted}

\section{Parseval'S Relations}

If $G[k]$ denotes the $N$ point DFT of the length $N$ sequence $g[n]$, then:

\begin{align}
    \sum_{n = 0}^{N - 1} \lvert g[n] \rvert^{2} =
    \dfrac{1}{N} \sum_{k = 0}^{N - 1} \lvert G[k] \rvert^{2}
\end{align}

The code below verifies the relation for a sequence $g[n]$:

\subsection{Python Implementation}

%python/parsevalsRelation.py%
\begin{minted}[breaklines, autogobble] {python}
    import numpy as np
    from scipy import fft

    g = np.concatenate([np.arange(128), np.arange(128, -1, -1)])

    LHS = np.sum(g ** 2)

    G = fft.fft(g)

    RHS = 1 / G.size * np.sum(np.abs(G) ** 2)

    print("LHS:")
    print(LHS)

    print("")

    print("RHS:")
    print(RHS)
\end{minted}

\paragraph{Output}

\begin{minted}[breaklines, autogobble] {text}
    LHS:
    1398144

    RHS:
    1398143.9999999993
\end{minted}

\section{Verifying Parseval'S Relation For a Sequence}

Generate a randoom complex sequence of $500$ values. Verify Parseval's relation
for the sequence.

\subsection{Python Implementation}

%python/verifyParsevalsRelation.py%
\begin{minted}[breaklines, autogobble] {python}
    import numpy as np
    from scipy import fft

    N = 500

    g = np.random.random_sample(N) + 1j * np.random.random_sample(N)

    LHS = np.sum(np.abs(g) ** 2)

    G = fft.fft(g)

    RHS = 1 / G.size * np.sum(np.abs(G) ** 2)

    print("LHS:")
    print(LHS)

    print("")

    print("RHS:")
    print(RHS)
\end{minted}

\paragraph{Output}

\begin{minted}[breaklines, autogobble] {text}
    LHS:
    340.555977018618

    RHS:
    340.555977018618
\end{minted}

\end{document}
