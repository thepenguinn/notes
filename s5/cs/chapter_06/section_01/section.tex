\documentclass[../../course]{subfiles}

\input{section_header.tex}

\begin{document}

\section{Introduction} \label{sec:ch06Intro}

Before we talk about \lt, let's ask ourselves a question, what the heck is even a
\textsc{transform}?

\paragraph{Transforms:} A \textsc{transform} is \emph{something} that
takes an \emph{input}\footnote{could be \emph{function}.} in a specific
\emph{domain} and spits out a \emph{function} in another \emph{domain}.
\lt is one such \textsc{transform} among many.

\paragraph{Laplace Transform:} Now what is \lt? And what does it do to our
\emph{input function}? \lt takes a \emph{function} in the \emph{time domain}
and spits out another \emph{function} that has the entire \textsc{s-plane} as
the \emph{domain}.

\paragraph{Mathematical Definition:} Mathematically \lt can be described as,

\begin{align}
    \mathcal{L}\{f(t)\} &= \mathcal{F}(s) \\
    &= \int_{0}^{\infty} f(t) e^{-st} dt
\end{align}

Where,

\begin{itemize} [label=]
    \item $\mathcal{L}\{f(t)\}$: is the \lt itself.

    \item $f(t)$: is the \emph{input function}.

    \item $s$: is a \emph{complex number} in the form $\alpha + j \omega$.

    \item
        Where,

        \begin{itemize} [label=]
            \item $\alpha$: is the \emph{real part}.
            \item $\omega$: is the \emph{imaginary part}.
        \end{itemize}
\end{itemize}


\end{document}
