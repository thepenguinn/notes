\documentclass[../../course]{subfiles}

\input{section_header.tex}

\begin{document}

\section{Taking DTFT of the Complex Sequences} \label{sec:wrkTakingDTFTCplxSeqs}

In the previous section, we have visualized the generated complex sequences. In this
section we will be taking DTFT for them.

\begin{align}
    X(e^{j \omega}) &= \sum_{n = 0}^{N - 1} x[n] e^{(-j \omega n)}
\end{align}

\subsection{Taking DTFT using Python}

%python/taking_dtft.py%
\begin{minted}[breaklines, autogobble, mathescape] {python}
    import numpy as np
    import matplotlib.pyplot as plt

    plt.figure(figsize = (6, 2.5))

    def dtft_factory(x_signal):

        length = len(x_signal)
        def dtft(freq):
            sum = 0
            for n in range(len(x_signal)):
                sum = sum + x_signal[n] * np.exp(-1j * 2 * np.pi * freq * n / length)

            return sum

        return dtft

    def sin_factory(freq):
        return lambda t: np.cos(2 * np.pi * freq * t)

    time = np.linspace(0, 1, 30)
    sin = sin_factory(1)
    sin_val = sin(time)

    sin_dtft = dtft_factory(sin_val)

    # plt.stem(time, sin_val)
    # plt.savefig("../plots/test.pdf")

    freq = np.linspace(0, 30, 30 * 30)

    freq_val = sin_dtft(freq)

    plt.stem(freq, freq_val)
    plt.savefig("../plots/test.pdf")

\end{minted}

\subsection {Taking Fourier Transform}

\begin{align}
    F(\xi) &= \int_{- \infty}^{\infty} f(x) e^{(- j 2 \pi \xi x)} dx
\end{align}

%python/taking_fourier.py%
\begin{minted}[breaklines, autogobble, mathescape] {python}
    import numpy as np
    import matplotlib.pyplot as plt
    from scipy.integrate import quad

    def ft_factory(x_fn, l_lim, u_lim):

        def gen_int_fn(xi):
            return lambda x: x_fn(x) * np.exp(-1j * 2 * np.pi * x * xi)

        def ft(xi):
            val = quad(gen_int_fn(xi), l_lim, u_lim, limlst = 1000)
            return val[0]

        return ft

    def sin(t):
        return (np.sin(2 * np.pi * 3 * t) + np.cos(2 * np.pi * 2 * t) + np.sin(2 * np.pi * 10 * t)) # + 1j * np.sin(2 * np.pi * 4 * t)

    freq = np.linspace(8, 20, 300)
    ft = ft_factory(sin, 0, 4)

    ft_val = np.ndarray(len(freq), dtype = np.cdouble)

    # ft_val[0] = (1, 1j)
    # print(ft_val[0])

    for i in range(len(freq)):
        ft_val[i] = ft(freq[i])

    # print(ft_val.real)

    fn_val = sin(freq)

    plt.subplot(3, 1, 1)
    plt.plot(freq, fn_val)

    plt.subplot(3, 1, 2)
    plt.plot(freq, ft_val.real)

    plt.subplot(3, 1, 3)
    plt.plot(freq, ft_val.imag)

    plt.savefig("../plots/ft.pdf")

\end{minted}

%python/proper_dtft.py%
\begin{minted}[breaklines, autogobble, mathescape] {python}
    import numpy as np
    import matplotlib.pyplot as plt

    plt.figure(figsize = (6, 2.5))

    def dtft_factory(x_signal):

        length = len(x_signal)
        def dtft(freq):
            sum = 0
            for n in range(len(x_signal)):
                sum = sum + x_signal[n] * np.exp(-1j * 2 * np.pi * freq * n / length)

            return sum

        return dtft

    # 0 * (1 / samp_freq)
    def disc_seq(samp_freq, freq):
        samp_period = 1 / samp_freq
        return lambda n: np.cos(2 * np.pi * freq * samp_period * n)

    # i dont get this, shit im an idiot
    seq = disc_seq(500, 30)
    seq_val = []
    for i in range(64):
        seq_val.append(seq(i))

    seq_val = np.array(seq_val)

    dtft = dtft_factory(seq_val)

    freq = np.linspace(0, 30, 31)

    dtft_val = dtft(freq)

    plt.stem(freq, dtft_val.real)
    plt.savefig("../plots/proper_dtft.pdf")

\end{minted}

%python/real_dft.py%
\begin{minted}[breaklines, autogobble, mathescape] {python}
    import numpy as np
    import matplotlib.pyplot as plt
    from scipy import fft

    plt.figure(figsize = (6, 2.5))

    def dft_factory(x_signal):
        length = len(x_signal)

        def dft(k):
            sum = 0

            for n, xn in enumerate(x_signal):
                sum = sum + xn * np.exp(-1j * 2 * np.pi * k * n / length)

            return sum

        return dft

    def sin_factory(freq):
        return lambda t: np.cos(2 * np.pi * freq * t)

    N = 500

    time = np.linspace(0, 1, N)
    sin = sin_factory(8)
    sin_val = sin(time)

    dft = dft_factory(sin_val)

    dft_val = []
    for i in range(N):
        dft_val.append(dft(i))
        if dft_val[i] > 4:
            print(i)


    plt.stem(dft_val)
    plt.savefig("../plots/delme.pdf")

\end{minted}

\paragraph{Output}

\begin{minted}[breaklines, autogobble] {text}
    7
    8
    492
    493
\end{minted}

%python/finally_dtft.py%
\begin{minted}[breaklines, autogobble, mathescape] {python}
    import numpy as np
    import matplotlib.pyplot as plt

    plt.figure(figsize = (6, 2.5))

    def dtft_factory(x_signal):

        def dtft(freq):

            sum = 0
            for n in range(len(x_signal)):
                sum = sum + x_signal[n] * np.exp(-1j * 2 * np.pi * n * freq)

            return sum

        return dtft

    def sin_factory(freq):
        return lambda t: np.cos(2 * np.pi * freq * t) + 1j * np.cos(2 * np.pi * 10 * freq * t)

    N = 1000

    time = np.linspace(0, 1, N)
    sin = sin_factory(4)
    sin_val = sin(time)

    dtft = dtft_factory(sin_val)

    freq = np.linspace(0, 1, 1000)

    final = np.ndarray(len(freq), dtype = np.cdouble)

    for i, f in enumerate(freq):

        final[i] = dtft(f)
        # print(final[i])
        if final[i].real > 10:
            print("Real", f * N)
        if final[i].imag > 10:
            print("Imaginary", f * N)

    # plt.plot(freq * N, final.imag)
    # plt.savefig("../plots/imdying.pdf")
\end{minted}

\paragraph{Output}

\begin{minted}[breaklines, autogobble] {text}
    Real 4.004004004004004
    Imaginary 40.04004004004004
    Imaginary 959.95995995996
    Real 995.9959959959959
\end{minted}

%python/test.py%
\begin{minted}[breaklines, autogobble, mathescape] {python}
    # stupid me.

    def gen(n):
        for i in range(n):
            yield i

    g = gen(3)

    print(list(g))


\end{minted}

\paragraph{Output}

\begin{minted}[breaklines, autogobble] {text}
    [0, 1, 2]
\end{minted}

\end{document}
