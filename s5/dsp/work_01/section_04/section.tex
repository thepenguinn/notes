\documentclass[../../course]{subfiles}

\renewcommand\thesection{\arabic{section}}


\begin{document}

\section{Taking DTFT of the Complex Sequences} \label{sec:wrkTakingDTFTCplxSeqs}

In the previous section, we have visualized the generated complex sequences. In this
section we will be taking DTFT for them.

\begin{align}
    X(\omega) &= \sum_{n = 0}^{N - 1} x[n] e^{(-j \omega n)}
\end{align}

\subsection{Taking DTFT using Python}

%python/taking_dtft.py%
\begin{minted}[breaklines, autogobble, mathescape] {python}
    import numpy as np
    import matplotlib.pyplot as plt

    # F = 28
    # t = np.linspace(0, np.pi / 16, 1000)
    #
    # x1 = np.sin( 2 * np.pi * F * t)
    # x2 = np.sin( 2 * np.pi * (F + 0.1) * t)
    # x3 = np.sin( 2 * np.pi * 2 * F * t)
    #
    # cplx_a = x1 + 1j * x1
    # cplx_b = x1 + 1j * x2
    # cplx_c = x1 + 1j * x3
    # cplx_f = x2 + 1j * x3

    def dtft_factory(x_signal):

        def dtft(omega):
            sum = 0
            for n in range(len(x_signal)):
                sum = sum + x_signal[n] * np.exp(-1j * omega * n)

            return sum

        return  dtft

    ## dtft = dtft_factory(cplx_a[0:10])
    # dtft = dtft_factory(np.array([0, 1, 2, 3, 4, 4, 4, 4, 3, 2, 1, 0], dtype = np.cdouble))
    time = np.linspace(0, 1, 1000)
    sin = np.sin(2 * np.pi * 3 * time) + np.sin(2 * np.pi * 1 * time)

    plt.subplot(3, 1, 1)
    plt.plot(time, sin)
    plt.grid()

    dtft = dtft_factory(sin)
    freq = np.linspace(0, 1, 1000)
    out = dtft(freq)

    plt.subplot(3, 1, 2)
    plt.plot(freq, out.real)
    plt.grid()

    plt.subplot(3, 1, 3)
    plt.plot(freq, out.imag)
    plt.grid()

    plt.savefig("../plots/test.pdf")
    # print(cplx_a)



    # A = fft.fft(cplx_a)

    # plt.subplot(2, 1, 1)
    # plt.plot(t, A.real)
    # plt.grid()
    #
    # plt.subplot(2, 1, 2)
    # plt.plot(t, A.imag)
    #
    # plt.savefig("../plots/test.pdf")
    #
    # print(numpy)

\end{minted}

\end{document}
