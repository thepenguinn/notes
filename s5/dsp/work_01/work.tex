\documentclass[../course]{subfiles}

\titleformat{\chapter}
{\normalfont\LARGE\bfseries}{\thechapter}{1em}{}
\titlespacing*{\chapter}
{0pt}{3.5ex plus 1ex minus .2ex}{2.3ex plus .2ex}

\renewcommand\thesection{\arabic{section}}


\def\author{Daniel V Mathew}
\def\supervisor{Prof. Premson Y}
\def\title{FREQUECNCY ANALYSIS OF SIGNALS}
\def\subtitle{DIGITAL SIGNAL PROCESSING}
\def\subsubtitle{SIMULATION ASSIGNMENT}

\def\authordesc {
    Roll No: 28 \\
    Semester: 5 \\
    Department of Electronics and Communication \\
    {\bfseries Rajiv Gandhi Institute of Technology} \\
}

\def\supervisordesc {
    Assistant Professor \\
    Department of Electronics and Communication \\
    {\bfseries Rajiv Gandhi Institute of Technology} \\
}

\def\abstractcontent {
    Generate $3$ frequencies, $X$, $2X$ and $2X.1$. Then find the DFT and DTFT of
    the combined signal and identify the frequency components. Try with $4$
    different sampling frequencies. One is $4X$, other one slightly greater than
    $4X$, one much greater than $4X$ and one much less than $4X$. For the DFT,
    select $N = 32$. In the sampled signal, with $N = 32$, zero pad with another 32
    zeroes and find the 64 point DFT and DTFT. Write your comments about frequency
    analysis from the results. $X$ will be your class number. Python or Matlab can
    be used for this assignment.
}

\begin{document}

\ifSubfilesClassLoaded {
    \ifundef{\author}{
    \def\author{Name of Author}
} {}

\ifundef{\supervisor}{
    \def\supervisor{Name of Supervisor}
} {}

\ifundef{\title}{
    \def\title{TITLE OF THE WORK}
} {}

\ifundef{\subtitle}{
    \def\subtitle{COURSE NAME}
} {}

\ifundef{\subsubtitle}{
    \def\subsubtitle{COURSE WORK TYPE}
} {}

\ifundef{\authordesc}{
    \def\authordesc{
        First line \\
        second line \\
        third line \\
    }
} {}

\ifundef{\supervisordesc}{
    \def\supervisordesc{
        First line \\
        second line \\
        third line \\
    }
} {}

\ifundef{\abstractcontent}{
    \def\abstractcontent{
        Say something about what this paper is about.
    }
} {}

\renewcommand{\maketitle} {
    \begin{center}
        {\LARGE \bfseries \subtitle}
        \vspace{50pt} \\
        {\bfseries \subsubtitle}
        \vspace{10pt}
    \end{center}

    \begin{center}
        \rule{\textwidth}{.75pt}
        \vspace{2pt}
    \end{center}

    \begin{center}
        {\huge \bfseries \title}
        \vspace{2pt}
    \end{center}

    \begin{center}
        \rule{\textwidth}{.75pt}
        \vspace{30pt}
    \end{center}

    \begin{center}
        \begin{minipage}{.86\textwidth}
            {\large
            \noindent
            \textit{Author}
            \hfill
            \textit{Supervisor}

            \vspace{4pt}

            \noindent
            {\bfseries \author}
            \vspace{2pt}
            \hfill
            {\bfseries \supervisor}

            \vspace{15pt}

            \noindent
            {\begin{minipage}[b] {0.4\textwidth}
                \vfill
                \raggedright
                {\small \itshape \authordesc}
            \end{minipage}
            \hfill
            \begin{minipage}[b] {0.4\textwidth}
                \vfill
                \raggedleft
                {\small \itshape \supervisordesc}
            \end{minipage}}

            }
        \end{minipage}
    \end{center}
}

\maketitle

\renewenvironment{abstract} {
    \vspace{30pt}
    \begin{center}
        {\large \bfseries \abstractname}
    \end{center}
    \vspace{4pt}
} {
}

\vspace{20pt}

\begin{abstract}
    \begin{center}
        \begin{minipage}{0.86\textwidth}
            \setlength{\parindent}{20pt}
            {\large

            \abstractcontent

            }
        \end{minipage}
    \end{center}
\end{abstract}

\vspace{80pt}
\begin{center}
    {\LARGE \bfseries
    \today
    }
\end{center}

} {
    \chapter{Frequecncy Analysis of Signals} \label{chp:wrkFrequencyAnalysis}
}

%Introduction%
\subfile{section_01/section.tex}

%Generating Sequences%
\subfile{section_02/section.tex}

\end{document}
