\documentclass[../../course]{subfiles}

\renewcommand\thesection{\arabic{section}}


\begin{document}

\section{Conclusion} \label{sec:conclusion}

In the previous sections we've generated a bunch of \emph{complex signals}. And
we sampled and we've analysed their \emph{frequency spectrum}. We took the
$32$ point \textsc{DTFT}s and \textsc{DFT}s. And we've also padded them with
$32$ more zeros and found the corresponding \textsc{DTFT}s and \textsc{DFT}s.
And found that for \textsc{DTFT}s, this padding is not that useful, but for
\textsc{DFT}s they seem to increase the \emph{resolution} of the \textsc{DFT}s.


And we've also saw the interesting graphical properties, of \textsc{DTFT}s and
\textsc{DFT}s.

\end{document}
