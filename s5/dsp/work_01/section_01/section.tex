\documentclass[../../course]{subfiles}

\renewcommand\thesection{\arabic{section}}


\begin{document}

\section{Introduction} \label{sec:wrkFrequencyAnalysisIntro}

Generate $3$ frequencies, $X$, $2X$ and $2X.1$. Then find the DFT and DTFT of
the combined signal and identify the frequency components. Try with $4$
different sampling frequencies. One is $4X$, other one slightly greater than
$4X$, one much greater than $4X$ and one much less than $4X$. For the DFT,
select $N = 32$. In the sampled signal, with $N = 32$, zero pad with another 32
zeroes and find the 64 point DFT and DTFT. Write your comments about frequency
analysis from the results. $X$ will be your class number. Python or Matlab can
be used for this assignment.

\begin{figure}
    \centering
    \adjustbox{max width = 0.7\textwidth} {
        \includegraphics[height = 0.8\textheight] {tikzpics/test.pdf}
    }
    \captionof{figure} {Testing Figure}
    \label{fig:testingFigure}
\end{figure}



\end{document}
