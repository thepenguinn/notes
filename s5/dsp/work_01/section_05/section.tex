\documentclass[../../course]{subfiles}

\input{section_header.tex}

\begin{document}

\section{Taking DFT of the Complex Sequences} \label{sec:wrkTakingDFTCplxSeqs}

In the previous Section, we took \textsc{DTFT}s of the \emph{sequences} we have
generated. In this Section, we will be taking \textsc{DFT}s of those same signals
and we can compare our findings. But before that let's take a look at what \textsc{DTF}s
are, and how to compute them.

\subsection{Discrete Fourier Transform}

Discrete Fourier Transforms or \textsc{DFT}s are just a special case of previously
mentioned \textsc{DTFT}s. \textsc{DTFT}s take a \emph{discrete} input sequence and
produces a \emph{continous} signal. In the case of \textsc{DFT}, they take \emph{discrete}
input but produces a signal in the \emph{discrete} form itself. In other words, they
if we \emph{sample} \textsc{DTFT} we will get \textsc{DFT}. Mathematically they can
be described as,

\begin{align}
    X[k] &= X(e^{j\omega}) |_{\omega = \frac{2 \pi k}{N}} \\
    &= \sum_{n = 0}^{N - 1} x[n] e^{-j w n} \bigg|_{\omega = \frac{2 \pi k}{N}} \\
    &= \sum_{n = 0}^{N - 1} x[n] e^{\big(-j \frac{2 \pi k}{N} n \big)} \label{eqn:dftK}
\end{align}

where,

\begin{itemize} [label=]
    \item $X[k]$: is the \textsc{DFT} itself.

        where,

        \begin{itemize} [label=]
            \item $k$: varies from $0$ to $N - 1$
        \end{itemize}

    \item $N$: is the total \emph{sample count}.
    \item $x[n]$: is the \emph{input sequence}.

\end{itemize}

\subsection{Implementing DTF using Python}

Just like we did with \textsc{DTFT}s let's implement a similar \mintinline{python}{dtf_factory}
that would take

%python/dft_factory.py%
\begin{minted}[breaklines, autogobble, mathescape] {python}
    import numpy as np

    def dft_factory(cplx_seq):

        sample_count = len(cplx_seq)
        def dft(k):
            sum = 0
            # see eq. ($\ref{eqn:dftK}$)
            for n, cplx in enumerate(cplx_seq):
                sum = sum + cplx * np.exp(-1j * 2 * np.pi * k * n / sample_count)
            return sum

        return dft
\end{minted}

\end{document}
