\documentclass[../../course]{subfiles}

\renewcommand\thesection{\arabic{section}}


\begin{document}

\section{Introduction} \label{sec:aCIntro}

Every \emph{communication systems} can be boiled down to a simple basic
structure, that includes a \emph{Transmitter}, a \emph{Receiver} and a
\emph{Channel}. And ofcourse, there's the \emph{information} that we are
sending and receiving. There are several ways to encode this information,
and strengthen\footnote{through, modulation and other techniques to make
the signal immune to noise and all.} the signal. Refer Figure \ref{fig:elemComSys}
for a overview of a typical \emph{communication system}.

\begin{figure}
    \centering
    \adjustbox{max width = 1\textwidth} {
        \includegraphics[height = 0.8\textheight] {tikzpics/epicElemComSys.pdf}
    }
    \captionof{figure} {Elements of a communication system}
    \label{fig:elemComSys}
\end{figure}




\end{document}
